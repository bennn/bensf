\documentclass[12pt]{article}

\usepackage[in]{fullpage}
\usepackage{multicol}
\usepackage{hyperref}

\newcommand{\mono}[1]{\textbf{\texttt{#1}}}

\newcommand{\hdr}[2]{\vspace{-0.4cm}{\flushleft{\hrulefill\\\textbf{#1}\hfill{#2}\\\vspace{-0.2cm}\hrulefill}}\vspace{0.1cm}}

\begin{document}
%%%%%%%%%%%%%%%%%%%%%%%%%%%%%%%%%%%%%%%%%%%%%%%%%%%%%%%%%%%%%%%%%%%%%%%%%%%%%%%%
%% Please outline your educational and professional development plans and career goals.
%% How do you envision graduate school preparing you for a career that allows you to contribute to expanding scientific understanding as well as broadly benefit society?
%% Page limit - 3 pages 

%% Describe your personal, educational and/or professional experiences that motivate your decision to pursue advanced study in science, technology, engineering or mathematics (STEM).
%% Include specific examples of any research and/or professional activities in which you have participated.
%% Present a concise description of the activities, highlight the results and discuss how these activities have prepared you to seek a graduate degree.
%% Specify your role in the activity including the extent to which you worked independently and/or as part of a team.
%% Describe the contributions of your activity to advancing knowledge in STEM fields as well as the potential for broader societal impacts (See Solicitation, Section VI, for more information about Broader Impacts).

%% NSF Fellows are expected to become globally engaged knowledge experts and leaders who can contribute significantly to research, education, and innovations in science and engineering.
%% The purpose of this statement is to demonstrate your potential to satisfy this requirement.
%% Your ideas and examples do not have to be confined necessarily to the discipline that you have chosen to pursue.
%%%%%%%%%%%%%%%%%%%%%%%%%%%%%%%%%%%%%%%%%%%%%%%%%%%%%%%%%%%%%%%%%%%%%%%%%%%%%%%%

\hdr{DRAFT: NSF Personal Statement}{Ben Greenman}

%% Cornell was a culture shock. Sad, grades down, drained by work as a custodian.
%% Luckily my search for affordable housing led the to a house of ten upperclassmen engineers.
%% They recommended I take classes in the one field they all enjoyed: computer science.
%% They were right.
%% I took CS, becaame a TA.

%% Early in CS career, joined a startup of 5 people \textemdash 2 entr. 1 broker 1 programmer \textemdash and helped them buid their business over the summer and part-time during school.
%% However Cornell CS taught met here's more to life than just computing the solution to the problem: want to compute well.
%% aka try research.
%% Unsure how to begin because of my nonstandard background, I thought I'd still go into industry.
%% But then, shortly after Professor Ross arrived on campus I sent himan email, we started working together, and the rest is history.

As I was growing up, my parents imposed a strict discipline on my life.
Looking back, I can see they wanted the best for me~\textemdash~both had associate degrees, and hoped I would earn a bachelor's~\textemdash~but by the time I entered high school I had resolved to move out when I turned 18.
To support myself, I worked a variety of evening jobs: first at a local Taqueria (where I eventually became a kitchen supervisor) and later on as a janitor.
It was during one of these late nights at the restaurant that one customer, a lawyer, told me about his experiences as a student at Cornell University.
That night I decided Cornell was my dream school, and applied to the Industrial and Labor Relations program the following year.

To my delight, Cornell accepted me as a guaranteed transfer\footnote{http://cornellsun.com/blog/2013/09/30/guest-room-inside-cornells-guarantee-transfer-system/} and I looked forward to arriving as a sophomore.
Until then, I spent a terrific freshman year at Hudson Valley Community College, taking a few courses to satisfy Cornell's requirements and also two semesters of calculus and physics.
I loved the welcoming, collegial environment at HVCC~\textemdash~my teachers were encouraging and helpful, and my fellow students were friendly and cared about learning.
The experience made me all the more excited about going to Cornell.

But when I finally arrived in Ithaca, I experienced a culture shock.
The only friendly students I met were those trying to recruit me into student government organizations, and teachers cared more about publishing articles than giving quality lectures and assignments.
It was a sad time.
My grades were poor, I felt disillusioned, and my evening work as a janitor left me drained.
Thankfully, my search for affordable housing had led me to an apartment of fun, upperclassmen engineers.\footnote{http://screwyoursnark.com/}
They encouraged me to try a minor in computer science.
Following their advice was one of the best decision I have ever made.

After finishing my second course in the CS department, I left my janitor job to TA functional programming.
The following summer, I got a job at a New York City startup consisting of two entrepreneurs, one real estate broker, and one programmer.
For the rest of my undergraduate career, I worked part-time both as a TA and as a software engineer for this company.
Though I was curious about research, I felt ashamed of my non-standard background and figured industry would be a better fit.
Then just before I graduated, Cornell hired Ross Tate and I decided to try approaching him.
The rest is history.

\hdr{Decidable Subtyping for Object-Oriented Langauges}{Summer~\textendash~Fall 2013}

Professor Ross Tate leads a double life.
He is also the type systems advisor for teams at Red Hat and JetBrains working to create object-oriented languages designed for managing large software projects.
These companies have three main priorities: they want code to be self-documenting, highly reusable, and backed with strong compile-time guarantees.
However, the infrastructure for reusable code sometimes caused the compiler to loop infinitely on useful programs.
In particular, a simple function for comparing two binary trees would cause an out-of-memory error before the code was even run~\textemdash~the compiler could not understand certain recursive definitions made by programmers.

Curious to learn if similar companies had encountered this issue, I implemented a source code analyzer to detect cycles in the inheritance hierarchy of a Java project.\footnote{https://github.com/bennn/inheritance-cycles}
Next I modified the \mono{javac} compiler to log a warning whenever a class/interfaces used in a cycle appeared as a field, type argument, method parameter, or return type.\footnote{https://github.com/bennn/javab}
After running this analysis on over 13 million lines of code taken from 60 open-source Java projects, Ross and I concluded that classes/interfaces used to create inheritance loops are used \emph{only} to make such loops.
Thus, Ross, myself, and his newly-acquired graduate student Fabian M\"uhlb\"ock spent the Fall creating a practical compile-time algorithm that resolved the issue~\cite{shapes}.
We presented our findings in Edinburgh this past summer, and there I was very proud to have thoughtfully solved a problem motivated by colleagues in industry.

\hdr{Conditional Inheritance}{Spring~\textendash~Summer 2014}

The following Spring, I continued to study object-oriented type systems.
With the issue of infinite loops solved, Ross and I hoped to build a mechanism for code reuse called conditional inheritance.
For example, conditional inheritance would let a programmer declare that a list is clonable if there is a procedure for making copies of all its elements.

At the time, we thought this extension would be straightforward.
Indeed, for a language like Java, where the range of a variable is specified at each use-site, such conditions are decidable in our system.
But the story for languages like C++, where a variable's definition is the only guide for its use, is much more complicated.
Week after week, I presented new designs to Ross only to return to the drawing board when one of us realized a corner case it missed.
After 4 months of iteration, we admitted that any fully backwards-compatible design would be too restrictive to be practically useful.
In light of this, I am now building a language and type-checker that allows conditional inheritance, but only on classes and interfaces that are uniquely determined by their parameters.
The goal is to test whether this more restricted strategy can still express necessary utilities like the Java collections library.

\hdr{Kleene Coalgebras}{Spring 2014}

Also that Spring, I worked with Professor Dexter Kozen and an exchange student from the University of Copenhagen, Ulrik Rassmussen, on coalgebraic decision procedures.
Much of Dexter's research is on practical applications of Kleene algebra, the language of iteration.
Two recent examples are NetKAT~\cite{anderson2014netkat}, a sound and complete network programming language, and KAT+B!~\cite{GKM14a}, which reasons about mutable boolean variables.
These theories provide \emph{algebraic} decision procedures useful for determining when two terminating programs are equivalent.
However, many useful programs, like user interfaces or servers, do not terminate but rather behave like automata.
A coalgebraic decision procedure allows reasoning about these ``endless'' programs.

When we began, Dexter and his collaborators had already developed coalgebraic procedures for KAT (Kleene Algebra with Tests) and NetKAT~\cite{foster2014coalgebraic}, so we focused on KAT+B!.
Together, Ulrik and I studied the past work on Kleene algebras and on coalgebras~\textemdash~along the way teaching each other some category theory and automata theory~\textemdash~and determined that we could construct the decision procedure for KAT+B! following very similar reasoning as had been done for KAT and NetKAT.
This prompted a further investigation into a unified system for finding the coalgebraic decision procedure of any Kleene algebra with additional equations, but this remains work-in-progress.
What I gained from this experience was a finer appreciation of theoretical techniques in programming langauges and of teamwork.
Neither myself nor Ulrik would have accomplished much if we had struggled on the problems alone.

\hdr{Broader Impacts}{}

Teaching has been a part of my life for nearly a decade.
For three years in Albany, I supervised elementary school students as a volunteer at my church's vacation bible school.
While a student at Hudson Valley, I volunteered at the local Boys and Girls club.
During my first summer in Ithaca, I created and taught a summer camp on LEGO Mindstorms robots.

On the other hand, learning computer science has been a thrill for me, so much that I have trouble containing my excitement when thinking about beautiful topics like type inference, induction-recursion, and hashing.
Being able to combine these two loves: teaching and functional programming, has been a dream come true.\footnote{As one student wrote on their course evaluations: ``I have to imagine that if functional programming were a person, [Ben] would want to marry it. It's adorable.''}
Indeed, the 5 semesters I spent teaching functional programming made my Cornell experience wonderful.

With this in mind, a big reason why I chose Northeastern University for graduate studies was the potential I saw for educational outreach.
Northeastern is home to the ProgramByDesign initiative, which for the past 10 years (20, if we go back to when PBD was known as TeachScheme!) has taught high school educators how to effectively teach algebra through functional programming.
This mission is the perfect fit for me.
As my academic history shows, I had an awful time trying to understand the seemingly arbitrary rules of mathematics until I learned functional programming.
At that point, everything ``clicked'' and I was able to become the person I am today.
Having gone through this experience puts me in a unique position to help others reach their own potential, and I am eagerly seeking opportunities to help with an upcoming ProgramByDesign workshop.

\vfill
\renewcommand{\section}[2]{}
\begin{multicols}{2}
\footnotesize
\bibliographystyle{plain}
\bibliography{nsf}
\end{multicols}

\end{document}
