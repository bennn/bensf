\documentclass[12pt]{article}

\usepackage[in]{fullpage}
\usepackage{multicol}
\usepackage{hyperref}

\newcommand{\mono}[1]{\textbf{\texttt{#1}}}

\newcommand{\hdr}[2]{\vspace{-0.4cm}{\flushleft{\hrulefill\\\textbf{#1}\hfill{#2}\\\vspace{-0.2cm}\hrulefill}}\vspace{0.1cm}}

\begin{document}
%%%%%%%%%%%%%%%%%%%%%%%%%%%%%%%%%%%%%%%%%%%%%%%%%%%%%%%%%%%%%%%%%%%%%%%%%%%%%%%%
%% Please outline your educational and professional development plans and career goals.
%% How do you envision graduate school preparing you for a career that allows you to contribute to expanding scientific understanding as well as broadly benefit society?
%% Page limit - 3 pages 

%% Describe your personal, educational and/or professional experiences that motivate your decision to pursue advanced study in science, technology, engineering or mathematics (STEM).
%% Include specific examples of any research and/or professional activities in which you have participated.
%% Present a concise description of the activities, highlight the results and discuss how these activities have prepared you to seek a graduate degree.
%% Specify your role in the activity including the extent to which you worked independently and/or as part of a team.
%% Describe the contributions of your activity to advancing knowledge in STEM fields as well as the potential for broader societal impacts (See Solicitation, Section VI, for more information about Broader Impacts).

%% NSF Fellows are expected to become globally engaged knowledge experts and leaders who can contribute significantly to research, education, and innovations in science and engineering.
%% The purpose of this statement is to demonstrate your potential to satisfy this requirement.
%% Your ideas and examples do not have to be confined necessarily to the discipline that you have chosen to pursue.
%%%%%%%%%%%%%%%%%%%%%%%%%%%%%%%%%%%%%%%%%%%%%%%%%%%%%%%%%%%%%%%%%%%%%%%%%%%%%%%%

\hdr{DRAFT: NSF Personal Statement}{Ben Greenman}

%% TODO probably want to section-ize
I fell in love with computer science when I took the introductory course on functional programming at Cornell University in Fall 2011.\footnote{\url{http://www.cs.cornell.edu/Courses/cs3110/2011fa/}}
Although I had taken programming classes before, this was my first experience with beautiful ideas like polymorphism, folding, type inference, and induction-recursion.
More importantly, this course taught me to engage with these ideas firsthand.
%% Unlike the classically-motivated math courses I had taken, which had also introduced beautiful ideas, in functional programming I could go home and build the abstract structures discussed in class.
Lectures on streams and circular lists became especially poignant after I had implemented operations on these ``infinite'' collections and watched as my programs looped indefinitely.
Indeed the entire course~\textemdash~up to the final punch line that all we had learned could be built from atoms called ``lambdas''\textemdash~was so exciting that I spent the next five semesters helping manage it as a teaching assistant.

%% During that time I wrote and graded numerous homework and exam questions, helped the course scale from 80 to 300 students as enrollment boomed, and above all else shared these beautiful and useful ideas to hundreds of fellow students.
%% Try as I might, I cannot think of a more rewarding way to have spent my free time.

Before I describe my research background, there are two ideas from this story that I want to explore further, both in this essay and in my future.
First is the importance of building things and second is educational outreach.

Constructing witnesses of the ideas I had seen on the blackboard was more than fun~\textemdash~it was essential to my learning.
Frankly, I do not fully understand a concept unless I have coded it up.
As an undergraduate, I was embarrassed to admit that; it seemed that other students and my professors simply thought faster than me, and that it was a bit shameful to need to write code.
But now I realize this thinking was completely wrong!
Coding \emph{is} understanding.
%% TODO a little awkwards, this sentence doesn't quite follow
%% Research ideas are of little value unless they are supported by real models and experiments.
A high-level knowledge of, say, a fully-verified operating system is nice to have, but the bottom line is whether one can build such a system and what we can learn from the artifact.
%% Consequently, the most influential research is that which has been built.
In research, the lesson is that theories mean little without supporting experiments; in life, I have learned that the best way to teach an idea (programming or otherwise) is to help another person build it.
%% I am grateful to have learned this lesson early in my career.

Second, I loved teaching and mentoring at Cornell; in fact, one student wrote a course evaluation saying: ``I have to imagine that if functional programming was a person, [Ben] would want to marry it. It's adorable.''.
At Northeastern, I am both continuing as university TA and getting involved with interdisciplinary outreach by teaching mathematics to middle and high-school students through the Bootstrap and ProgramByDesign initiatives.\footnote{Bootstrap: \url{http://www.bootstrapworld.org} \qquad ProgramByDesign: \url{http://www.programbydesign.org/}}
Bootstrap offers afterschool programs where students learn algebra by programming (again, coding is understanding).
I will be involved with this program because Matthias Felleisen, whose educational outreach over the past 20 years laid the groundwork for the bootstrap program, is one of my two faculty advisors and Emmanuel Schanzer, founder of the Bootstrap program, is currently earning his Ph.D in Mathematics Education at Harvard.

%%%%

Back to my undergraduate experience, for nearly two years I was quite happy taking courses, teaching functional programming, and meanwhile working part-time as a software engineer at an early-stage startup.
After graduation, I had planned to work full-time with this startup; however, I was not entirely confident in this decision.

%% #practice2theory
My first priority as an undergraduate had been to learn how to code.
I wanted to acquire a skill; I wanted control over my future.
But as I learned computer science, I realized there is more to life than just computing the answer to a problem.
Equally important is the notion of computing \emph{well}: optimizing some combination of time, space, keystrokes, or portability.
Looking back on the thousands of lines of code I had written as an employee, I realized that my future at industry would center around a very narrow interpretation of ``good computing''.
So I instead chose to do a Masters of Engineering at Cornell, began doing research with Professor Ross Tate, and have since been living the happily ever after.


\hdr{Decidable Subtyping for Object-Oriented Langauges}{Summer~\textendash~Fall 2013}

My first research project with Ross was an investigation into object-oriented type systems.
Before we met, he had been talking with professionals at Red Hat and JetBrains and was currently investigating how to define operations like equality and ordering on arbitrary data structures.
This problem is easily overlooked; for example, consider equality.
Any Java programmer will tell you that checking if two objects are equal is simple: just call the \mono{equals} method baked in to every object.
However, this approach requires run-time type casts, does not scale to other operations, and completely ignores the fact that equality is not defined on functions.
Furthermore it allows expressions like \mono{42.equals(`hello')} to pass type-checking instead of being flagged as errors.
Even worse, straightforward attempts to fix this issue lead to undecidable subtyping judgments because of the combination of variance and F-bounded polymorphism provided by mainstream languages.

When Ross first explained this project to me, I was blown away.
After just a few minutes talking about lists and equality, we had a 12-line Java program that caused the \mono{javac} and Eclipse compilers to crash with a stack overflow.
Curious to learn why programmers did not encounter similar issues in practice, I implemented a source code analyzer that would detect cycles in the inheritance hierarchy of a Java project.\footnote{\url{https://github.com/bennn/inheritance-cycles}}
Then to determine how these cycles were used, I modified the \mono{javac} compiler to log a warning whenever a class/interfaces used to complete a cycle described a field, type argument, method parameter, or return type.\footnote{\url{https://github.com/bennn/javab}}
After running this analysis on over 13 million lines of code taken from 60 open-source Java projects, we concluded that classes/interfaces like Java's \mono{Equatable}, which create inheritance loops, are used \emph{only} to make such loops.
Thus, Ross, myself, and his newly-acquired graduate student spent the Fall creating a practical algorithm for decidable subtyping that leveraged this observation~\cite{shapes}.
We presented our findings in Edinburgh this summer, and I was very proud that we had thoughtfully solved a problem motivated by colleagues in industry.

\hdr{Conditional Inheritance}{Spring 2014}

The following Spring, I continued this investigation of object-oriented type systems.
Although our PLDI paper had addressed decidable subtyping, we had not yet solved the original question of providing type-safe equality on arbitrary data structures!

At the time, the answer seemed straightforward.
We needed conditional inheritance to express properties like the definition of comparable lists: a list may be compared to other lists provided both lists are the same type and the elements inside the lists may be compared.
Indeed, for a language like Java, with wildcards and use-site variance, such conditions are decidable in our system.
However, we did not want to limit our results to use-site variance, so I explored decidable conditional inheritance for languages like C$^{\#}$ and Scala that use declaration-site variance.
This task proved difficult.
Week after week, I would present a design to Ross.
Within a few days one of us would realize a corner-case it missed.
After 4 months of iteration, we admitted that any correct design along these lines will be too restrictive to be practically useful.
In light of this, I am now building a language and type-checker that allows conditional inheritance, but only on invariant classes and interfaces.
The goal is to test whether this more restricted strategy can still express necessary utilities like the Java and Scala Collections libraries.

%% Thankfully scientists have the luxury of experimenting and occasionally failing.
%% Although my Spring semester with Ross did not end in a publishable finding, I still learned a great deal about both the foundations of object-oriented programming and the problems uet unsolved.

\hdr{Kleene Coalgebras}{Spring 2014}

Also in Spring 2013, I worked with Professor Dexter Kozen and an exchange student from the University of Copenhagen, Ulrik Rassmussen, on coalgebraic decision procedures.
Much of Dexter's research is on practical applications of Kleene algebra, the language of iteration.
Two recent examples are NetKAT~\cite{anderson2014netkat}, a sound and complete network programming language, and KAT+B!~\cite{GKM14a}, which adds mutable boolean variables to a Kleene algebra.
These theories provide algebraic decision procedures useful for determining when two terminating programs are equivalent.
However, many useful programs, like user interfaces or servers, do not terminate but rather behave like automata.
A coalgebraic decision procedure allows reasoning about these ``endless'' programs.

When we began, Dexter and his collaborators had already developed coalgebraic procedures for KAT (Kleene Algebra with Tests) and NetKAT~\cite{foster2014coalgebraic}, so we focused on KAT+B!.
Together, Ulrik and I studied the past work on Kleene algebras and on coalgebras~\textemdash~along the way learning some category theory and automata theory~\textemdash~and determined that we could construct the decision procedure for KAT+B! following very similar reasoning as had been done for KAT and NetKAT.
This prompted a further investigation into a unified system for finding the coalgebraic decision procedure of any Kleene algebra with additional equations, but unfortunately this remains work-in-progress.
What I gained from this experience was a finer appreciation of the problems being solved in theoretical programming languages research, and an opportunity to collaborate with a student who lives outside the United States.

\vfill
\renewcommand{\section}[2]{}
\begin{multicols}{2}
\footnotesize
\bibliographystyle{plain}
\bibliography{nsf}
\end{multicols}

\end{document}
