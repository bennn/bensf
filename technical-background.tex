\documentclass{article}

\usepackage{fullpage}
\usepackage{palatino}

\newcommand{\mono}[1]{\textbf{\texttt{#1}}}

\renewcommand\maketitle{
{\flushleft{\textsf{%
Ben Greenman
\hfill
DRAFT: NSF Technical Background
}}\\
\hrulefill}
}
\begin{document}
\maketitle

In April 2013, near the end of my final semester as an undergraduate student at Cornell, I found myself in an interesting position.
The startup I had been working with for the past year, and had intended to work for after I graduated, had just called and given me a formal, full-time offer.
So far so good, but they could not start paying me until the following October, after they finished a first round of fund raising.

Despite this discouraging news, I might still have accepted their offer.
It was a small and promising company; the five of us were like a family and held one another to high standards.
%% I was proud of the work I had designed and implemented for the startup, and had a lot to look forward to as a core member of the growing company.
Creating software at a fast-paced startup seemed, at the time, the best way to leverage the programming skills I had learned as an undergraduate so to grow as an individual and make an impact on the world.

That said, I had a few doubts about working in industry.
As much as I enjoyed the diverse work and fast pace, I felt there must be more to life than fine-tuning database queries, using a unit test suite as the final word on correctness, and leaving work promptly at 5:00pm for happy hour at a local bar.
We were doing things well but we were not creating masterpieces.
In retrospect, I am ashamed I did not fully realize this \emph{until} hearing the details of my offer, but I came to my senses, realized that life is too short for ugly things, knocked on Ross's door, and haven't looked back since.

\vspace{0.5cm}

That summer and the following year I completed a masters at Cornell and worked with Ross on a topic he had been thinking about since his time at UCSD: how to safely define operations like equality and comparison on parameterized classes in an object-oriented type system.
This is a serious issue that is easily overlooked.
For example, consider equality.
Any Java programmer will tell you that checking if two objects are equal is trivial: just call the \mono{.equals()} method baked in to every object.
However, this approach requires run-time type casts, does not scale to other operations, and completely ignores the fact that equality is not defined on many mathematical structures.
Furthermore it allows expressions like \mono{42.equals(``hello world'')} to pass type-checking, which should really be flagged as type errors.
One might try to fix this issue by declaring that a data structure is equatable if and only if its elements are equatable, but because mainstream languages support variance and F-bounded polymorphism, subtyping with these types is undecidable.

When Ross first explained this project to me, I was blown away.
After just a few minutes talking about lists and equality, we had a 12-line Java program that caused the \mono{javac} and Eclipse compilers to crash with a stack overflow.
At the time I was curious why programmers did not encouter decidability issues in the wild.
I had never been warned, either at school or in industry, to avoid certain patterns because they would break the compiler.
Thus I spent that summer analyzing current practices to determine how programmers actually defined and used types similar to equatable in practice.
To this end, I implemented a source code analyzer that would read the files in a Java project, create an XML document describing the supertypes and type parameters of each class or interface, and finally use Johnson's cycle-finding algorithm~\cite{johnson} to detect loops in the inheritance hierarchy.\footnote{https://github.com/bennn/inheritance-cycles}
Finally, to determine how these types were used throughout a project, I modified the \mono{javac} compiler to log a warning where a type like equatable was used as a field, type argument, method parameter, or return type.\footnote{https://github.com/bennn/javab}

After I ran this analysis on over 13 million lines of code taken from 60 open-source Java projects, Ross, myself, and his newly-acquired graduate student spent the Fall semester formalizing and writing up these findings.
Together with Ross's student, I worked on formalizing a set of rules for inheritance and proving that these rules ensured decidable subtyping.
Independently, I explored related work to determine how our observation fit with research at large, and wrote more drafts of the paper than I care to recall.
Each time we met, Ross would offer guidance and advice; sure enough, each week my work improved (or rather, each day, as the deadline loomed) until we had converged on a product that all three of us were proud of~\cite{shapes}.
Unlike the code I had written in industry, our paper was a finely-tuned and accessible piece of work.
Moreover, it represented a combination of ingenuity, engineering, and formal argument that I will apply to every future problem I work on.
Anything less would be an insult to myself and to my community.

\newpage

The following Spring, I continued to do research under Ross's guidance.
Our paper had laid a foundation for decidable subtyping that agreed with widespread industry practice, but aside from computable joins we had not yet built any features leveraging this type system.
In particular, we had not solved the original question of providing type-safe equality on arbitrary parameterized types!
My goal for the Spring was then to leverage decidable subtyping to solve this issue.
We wanted to be able to declare, for example, a list of objects such that the list was equatable to other lists if its elements were equatable to one another, and otherwise not.

This behavior is more formally known as conditional inheritance, and was in fact studied before by Wehr et~al. in their work on JavaGI~\cite{javagi}.
However JavaGI does not extend to object-oriented languages in general because it relies on the form of bounded existentials provided by wildcards to resolve ambiguous signatures.
Languages like C\# and Scala, which use declaration-site variance, do not have this power.
Nonetheless, I believed we could still build a system for expressive conditional inheritance, thereby formalizing properties like equality and clone-ability within the type system, rather than determining them in an ad-hoc manner at run time.

To cut a long story short, I ran into serious trouble with expressiveness.
Each set of type checking and method checking rules I wrote failed to catch some reasonable behavior.
As an example, if an invariant array class is the subtype of a covariant list interface, we can get unbound method errors unless all array or list elements uniformly support the same characteristics.
After each such disappointment, I would try to find the root of the problem, add the least-restrictive axiom or typing rule to the system, and begin critiquing anew.
Thanks to Ross's incredible knack for finding counterexamples, I never lingered on a single design.
Rather I spent a long time attempting to balance expressiveness with usefulness and reading foundational papers on the theories developed to explain object-oriented programming.
My conclusion is that our understanding of objects is fundamentally wrong.
That world grew like a jungle out in industry, where languages with objects proved essential for program organization, abstraction, and code reuse.
Meanwhile theoretical computer scientists were preoccupied working on purely functional languages or simple imperative languages.
Unlike objects, these academic efforts were built from first principles and connected the theories of types and programming languages to work done in category theory and in proof theory.
Also unlike objects, they were done right.
I fully sympathize with the contemporary opinion that objects were a fad.
The fact is, that opinion was wrong.
Objects matter; unfortunately, the fundamental reasons why they matter are difficult to uncover because ``objects'' as we know them from an ALGOL-like language combine a variety of programming concepts.
Some behave like ADTs, others behave like automata, and still others are best described as abstract message-passing servers.
At any rate, the conventional wisdom that an object is a self-referencing record, or just a collection of data with methods to access it, seems insufficient.

Thankfully we have the luxury, in academia, of experimenting and occasionally failing.
Although my Spring semester with Ross did not end in a publishable finding,\footnote{Dear advisors: I have a 12-page draft of the POPL paper. It's broken, but I could include it as a reference.} I learned a great deal.
Above all else, I am excited to learn more and will continue to pursue this goal regardless of where life takes me.

\bibliographystyle{plain}
\bibliography{nsf}

\end{document}
