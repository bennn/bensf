\documentclass{article}

\usepackage{fullpage}
\usepackage{palatino}

\newcommand{\mono}[1]{\textbf{\texttt{#1}}}

\renewcommand\maketitle{
{\flushleft{\textsf{%
Ben Greenman
\hfill
DRAFT: NSF Motivation
}}\\
\hrulefill}
}
\begin{document}
\maketitle

I fell in love with computer science when I took the introductory course on functional programming at Cornell University in Fall 2011.\footnote{http://www.cs.cornell.edu/Courses/cs3110/2011fa/}
Although I had taken programming courses before, this was my first real experience with computer science, and I loved every minute of it.

My first impression was that this class was ridiculously difficult.
Topics like pattern-matching, folding, and substitution came in hard and fast during the first three lectures, and meanwhile we learned the OCaml language from the ground up.
But the need for this fast pace soon became apparent: there was a wealth of material to talk about!
As the course progressed, I learned the beauty of ideas like closures, type inference, polymorphism, and induction-recursion.
More importantly, I engaged with these ideas firsthand!
Unlike the classically-motivated math courses I had taken, which had also introduced beautiful ideas, in functional programming I could actually build and engage with the abstract structures discussed in class.
Lectures on streams and circular lists became especially poignant after I had gone home and implemented operations on these ``infinite'' collections.
To this day it remains true that I only fully understand an idea once I have coded it up.

We spent the entire semester building up to more detailed and specialized ideas.
But then, in the final week of the course, we stopped abruptly and hit the punch line: everything procedure and every data structure we had written could be implemented in a minimal system called the $\lambda$-calculus.
Amazing! Tell me more!
Unfortunately, the semester was over.
But I took the obvious next step and enrolled as a teaching assistant for this same course on functional programming the following semester.
For the next five semesters, I served as a TA for the functional programming course.
During that time I wrote and graded numerous homework and exam questions, helped the course scale from 80 to 300 students as enrollment boomed, and above all else shared these beautiful and useful ideas to hundreds of fellow students.
Try as I might, I cannot think of a more rewarding way to have spent my free time.

\vspace{0.5cm}

There are two ideas from this story that I want to explore further, both in this essay and in my future.
First is this notion that building things really matters, and second is educational outreach.

Earlier, I noted that I personally did not fully understand concepts until I had coded them up.
It took about a year after finishing this course in functional programming, but I eventually learned that this was a widely-held opinion.
As it turned out, a man named Brouwer had been yelling about this same idea since the 1930's: that there is an enormous difference between specifying something and actually constructing it.
Just as one should be skeptical of an idea for revolutionary new operating system that has not yet been implemented, an assertion without a constructive proof is stuck in the world of fantasy (or at least, the world augmented with a few axioms).
Without going into further details about the connection between programs and proofs, I want to say that this correspondence is the biggest motivation I have for working in computer science.
Mathematicians are fond of saying that math shows up everywhere in nature, and that fundamental concepts in mathematics are immediately useful to every person on the Earth.
But this is not entirely accurate.
If I were to apply classical reasoning to social situations, I might infer the statement ``I don't dislike my sister-in-law'' to mean something entirely different from what the speaker intended.
On the other hand, intuitionistic logic and computer science really do have immediate consequences.
Basic computer skills help in nearly any profession, if nothing else then for personal organization, and any child old enough to talk needs to be aware that just because Aunt Cheryl ``doesn't dislike'' Aunt Margarent, it is not the case that the two like one another.
Toy examples aside, it's no stretch to claim that work done in computer science is going to have the single greatest influence on mankind in the next hundred years.
I want to be a part of that progress.

\vspace{0.5cm}

As for educational outreach, for the past two years I have loved sharing ideas in computer science.
Some of my best memories are from times I spent in the computer lab with students, presenting my opinion on topics and exploring their implications together.
The enthusiasm I have for the subject must have been infectious and apparent.
In one of my TA evaluations, I even received a comment from a student: ``I have to imagine that if functional programming was a person, he would want to marry it. It's adorable.''
Anyway, as much as I want to push boundaries and do research in computer science, I also want to share these ideas with others who do not know or have just been introduced to computer science.
This is why I chose Northeastern University for my graduate studies.
More than any other instituition I am aware of, Northeastern has faculty that care about teaching.
Chief among them is Matthias Felleisen, who in 1995 founded the TeachScheme! campaign to bring programming to high schools and middle schools across the country.
Over the following decade, Professor Felleisen and his groups organized a series of events and workshops to show teachers how to effectively use functional programming to engage students in mathematics and to teach design principles.
More recently, the program has offered after-school programs to children in inner-city school districts throughout Boston, Chicago, Maryland, and New York City.\footnote{http://www.bootstrapworld.org/}
This outreach is an inspiration to me, and I hope to get involved soon.

\end{document}
