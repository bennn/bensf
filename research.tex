\documentclass[12pt]{article}

\usepackage[in]{fullpage}
\usepackage{multicol}

\newcommand{\mono}[1]{\textbf{\texttt{#1}}}

\newcommand{\hdr}[2]{\vspace{-0.4cm}{\flushleft{\hrulefill\\\textbf{#1}\hfill{#2}\\\vspace{-0.2cm}\hrulefill}}\vspace{0.1cm}}
\begin{document}

\hdr{DRAFT: NSF Research Proposal}{Ben Greenman}

Developers creating a software project have a wide range of languages to choose from.
They pick the one best-suited to meet their immediate needs, but must often incorporate other languages as the project grows.
For example, Facebook began with a PHP script, but the now multi-billion dollar website currently includes code from at least 6 other languages, including C and Erlang.\footnote{http://www.quora.com/What-programming-languages-are-used-at-Facebook}
%% Similarly, Twitter has recently migrated its entire codebase from Ruby scripts to a mix of Scala and Java.\footnote{http://readwrite.com/2011/07/06/twitter-java-scala}
Conversely, Jane Street Capital was mostly Visual Basic and Excel before one aspiring employee began rewriting the core trading software in OCaml.\footnote{http://www.pcworld.idg.com.au/article/547567/won\_t\_believe\_what\_programming\_language\_wall\_street\_firm\_uses/}
%% Unfortunately, the interplay between these languages is completely unverified.
%% There is no formal notion of correctness between, say, OCaml code and C code.
This phenomena is widespread, and it arises because software projects encompass a spectrum of \emph{changing} goals but modern languages only succeed at a few specific tasks.
%% and this spectrum is more than any modern language can address.
%% Unfortunately, there are few tools available for reasoning about such multi-language systems.

My goal is to build a \emph{full spectrum programming language} that can accomodate a range of tasks, from high-assurance kernel software to dynamic scripts.
More importantly, this language will guarantee the interaction of its component languages.
Type system invariants will be preserved despite interaction with untyped modules, and scripts that grow into full-fledged programs will be straightforward to port into typed or even dependently typed code.

\hdr{Research Proposal}{}

This is not a strictly novel idea; in fact, the research community has been verifying small multi-language systems for the past decade.
Pioneering work by Matthews and Findler studied a combined language of Scheme and ML~\cite{matthews2007operational}, and more recently Osera et.~al examined the interaction of simple and dependent types~\cite{osera}.
In addition, other projects have explored topics like certified foreign-function interfaces~\cite{furr2005checking} and safe interoperability of untyped code with Java~\cite{gray2005fine}.
The time has come for a full-spectrum language that unifies these efforts.

My proposal is to study the interaction between dependently-typed and (un)typed code, building on the gradual typing work done by Sam Tobin-Hochstadt while at Northeastern University~\cite{tobin2010typed}.
Sam gave a full proof-of-concept for safe transitions between untyped and simply-typed modules.
I plan to extend his system with a dependently-typed language that permits all the formal verification of Coq or Agda, but moreover ensures safe operation with unverified languages similar to ML or C.
The feedback loop for realizing this plan will be to design a theoretical framework~\textemdash~initially I plan to use contracts to encode invariants like Coq proofs or parametricity requirements~\textemdash~and then implement it within the Racket ecosystem, which offers an extensible core language and diverse libraries to translate.

My journey will be guided by the experienced faculty at Northeastern and around Boston.
On campus, I have Matthias Felleisen, Amal Ahmed, and Jan Vitek: masters in their respective areas of design, theory, and practicality.
Across the river are Harvard's Greg Morrisett, who has been compiling types for over a decade, and MIT's Adam Chlipala, the US expert on programming with dependent types.
My training as an undergraduate in Industrial and Labor relations prepared me to bridge ideaologies and boundaries, and answering the question of full spectrum languages will require putting these lessons to use.

\newpage

\hdr{Broader Impacts}{}

The two motivating examples listed in my introduction are interesting, but relatively benign.
Much more compelling reasons for building a full spectrum language are the Swedish Pension system, which consists of 320,000 lines of Perl,\footnote{https://www.cs.purdue.edu/homes/jv/talks/dls09.pdf} and the HACMS project, which seeks to provide a simple interface for programming Unmanned Autonomous Vehicles (UAVs).\footnote{http://www.darpa.mil/Our\_Work/I2O/Programs/High-Assurance\_Cyber\_Military\_Systems\_(HACMS).aspx}

The pension system grew from a script that saved the country when a contractor failed to deliver, but now represents the absolute worst in legacy code.
Sweden desperately needs a plan to certify individual modules and protect its retirees because a brute-force rewrite is not an option.
On the other hand, the High-Assurance Cyber Military Systems (HACMS) project is seeking an answer to the opposite problem.
To create a simple API to their highly-sensitive machines, they essentially need a method of linking scripts to dependently typed programs.
These are precisely the issues my research will investigate.

That said, a full spectrum language is useful for more than large-scale software projects.
It is also the ideal language for teaching programming concepts to novices~\textemdash~either young students or adults seeking to learn a highly competitive skill~\textemdash~because it lets a teacher introduce new concepts without needing to teach a new language.
As a student of Matthias Felleisen, I will be able to introduce my language to high-school and college students learning to code via the ProgramByDesign initiative, and also leverage it as a tool to improve middle-school mathematics education through the Bootstrap program.

Bootstrap is an afterschool program developed by Emmanuel Schanzer, a Mathematics Education Ph.D student at Harvard, that teaches algebra to children in underpriviledged neighborhoods throughout New York, Massachusetts, Chicago, Maryland, and Florida.\footnote{http://www.bootstrapworld.org}
The cirriculum now uses a simple functional language to teach, but this raises issues because mathematical functions like factorial are undefined on particular, well-typed inputs.
By working with Bootstrap I hope to improve the quality of learning for a younger generation while simultaneously getting feedback on a tool, the full spectrum language, that has applications in every area from web design to UAVs to retirement management.
%% It's certainly crucial to my retirement plan!

\renewcommand{\section}[2]{}
\begin{multicols}{2}
\footnotesize
\bibliographystyle{plain}
\bibliography{nsf}
\end{multicols}

\end{document}
