\documentclass[12pt]{article}

\usepackage[in]{fullpage}
\usepackage{multicol}
\usepackage{hyperref}

\newcommand{\mono}[1]{\textbf{\texttt{#1}}}

\newcommand{\hdr}[2]{\vspace{-0.4cm}{\flushleft{\hrulefill\\\textbf{#1}\hfill{#2}\\\vspace{-0.2cm}\hrulefill}}\vspace{0.1cm}}
\begin{document}

\hdr{DRAFT: NSF Research Proposal}{Ben Greenman}

%% TODO new first sentence
%% Developers creating a software project have a wide range of languages to choose from.
The first step in designing software is to pick the language best suited for the task.
However, a single programming language is rarely sufficient to meet the needs of a growing project.
%% They pick the one best-suited to meet their immediate needs, but must often incorporate other languages as the project grows.
For example, Facebook began with a PHP script, but the now multi-billion dollar website currently includes code from at least 6 other languages, including C and Erlang.\footnote{http://www.quora.com/What-programming-languages-are-used-at-Facebook}
%% Similarly, Twitter has recently migrated its entire codebase from Ruby scripts to a mix of Scala and Java.\footnote{http://readwrite.com/2011/07/06/twitter-java-scala}
Conversely, Jane Street Capital was mostly Visual Basic and Excel before one aspiring employee began rewriting the core trading software in OCaml.\footnote{http://www.pcworld.idg.com.au/article/547567/won\_t\_believe\_what\_programming\_language\_wall\_street\_firm\_uses/}
%% Unfortunately, the interplay between these languages is completely unverified.
%% There is no formal notion of correctness between, say, OCaml code and C code.
This phenomena is widespread, and it arises because software projects encompass a spectrum of \emph{changing} goals but modern languages are designed to solve only one specific class of problems.
%% and this spectrum is more than any modern language can address.
%% Unfortunately, there are few tools available for reasoning about such multi-language systems.

My goal is to build a \emph{full spectrum programming language} that can accomodate a range of tasks, from high-assurance kernel software to dynamic scripts.
More importantly, this language will guarantee the interaction of its component languages.
Type system invariants will be preserved despite interaction with untyped modules, and scripts that grow into full-fledged programs will be straightforward to port into typed or even dependently typed code.

\hdr{Research Proposal}{}

This is not a strictly novel idea; in fact, the research community has been verifying small multi-language systems for the past decade.
Pioneering work by Matthews and Findler studied a combined language of Scheme and ML~\cite{matthews2007operational}, and more recently Osera et.~al examined the interaction of simple and dependent types~\cite{osera}.
In addition, other projects have explored topics like certified foreign-function interfaces~\cite{furr2005checking} and safe interoperability of untyped code with Java~\cite{gray2005fine}.
The time has come for a full-spectrum language that unifies these efforts.

My proposal is to study the interaction between dependently-typed and (un)typed code, building on the gradual typing work done by Sam Tobin-Hochstadt while at Northeastern University~\cite{tobin2010typed}.
Sam gave a full proof-of-concept for safe transitions between untyped and simply-typed modules.
I plan to extend his system with a dependently-typed language that permits all the formal verification of Coq or Agda, but moreover ensures safe operation with unverified languages similar to ML or C.
The feedback loop for realizing this plan will be to design a theoretical framework~\textemdash~initially I plan to use contracts to encode invariants like Coq proofs or parametricity requirements~\textemdash~and then implement it within the Racket ecosystem, which offers an extensible core language and diverse libraries to translate.

My progress will be directed by the huge group of experienced faculty at Northeastern University.
Matthias Felleisen has literally written the book on program design, and has spent the past seven years working on Typed Racket.
Amal Ahmed perfected the proof method of step-indexed logical relations and has been using them to prove groundbreaking results on single and multi language systems.
Jan Vitek, the voice of reason, has been actively benchmarking gradually-typed languages and bridging the gap between academia and industry.
I am extremely lucky to have found such diverse mentors at one institution.

\newpage

\hdr{Broader Impacts}{}

The two motivating examples listed in my introduction are interesting, but relatively benign.
Much more compelling reasons for building a full spectrum language are the Swedish Pension system, which consists of 320,000 lines of Perl,\footnote{https://www.cs.purdue.edu/homes/jv/talks/dls09.pdf} and the HACMS project, which seeks to provide a simple interface for programming Unmanned Autonomous Vehicles (UAVs).\footnote{http://www.darpa.mil/Our\_Work/I2O/Programs/High-Assurance\_Cyber\_Military\_Systems\_(HACMS).aspx}

The pension system grew from a script that saved the country when a contractor failed to deliver, but now represents the absolute worst in legacy code.
Sweden desperately needs a plan to certify individual modules and protect its retirees because a brute-force rewrite is not an option.
Conversely, the High-Assurance Cyber Military Systems (HACMS) project is seeking a solution for certified de-classification.
To create a simple API to their highly-sensitive machines, they essentially need a method of linking scripts to dependently typed programs.
These are precisely the issues my research will investigate.

Beyond its applications to real-world systems, a full-spectrum language is ideal for teaching programming concepts to novices.
Students currently need to learn a new language to experience new programming paradigms, but in the context of a full-spectrum language they can study these concepts and the connections between them without needing to learn a new syntax.
This has great potential for adults looking to change careers but do not have the time or resources to enroll in a full-time university program.

To that end, I hope to introduce my full spectrum language to the Bootstrap program, an afterschool initiative with strong ties to Northeastern that teaches algebra to middle school students.\footnote{http://www.bootstrapworld.org}
For one, many mathematical functions like factorial and median are undefined on certain well-typed inputs.
A simple functional language is not enough to accurately describe these functions~\textemdash~Bootstrap needs dependent types.
Second, dependent types allow extremely powerful reasoning about programs, enough for students to write lemmas and theorems they might find in a textbook.
Being able to rewrite theorems as programs will help students better understand the words in their algebra books and improve their formal reasoning skills.
Thus a full-spectrum language is really the key to a fully comprehensive program for teaching mathematics.

\vspace{-0.2cm}
\renewcommand{\section}[2]{}
\begin{multicols}{2}
\footnotesize
\bibliographystyle{plain}
\bibliography{nsf}
\end{multicols}

\end{document}
