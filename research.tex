\documentclass[12pt]{article}

\usepackage[in]{fullpage}
\usepackage{multicol}
\usepackage{hyperref}

\newcommand{\hdr}[2]{\vspace{-0.4cm}{\flushleft{\hrulefill\\\textbf{#1}\hfill{#2}\\\vspace{-0.2cm}\hrulefill}}\vspace{0.1cm}}
\begin{document}
%%%%%%%%%%%%%%%%%%%%%%%%%%%%%%%%%%%%%%%%%%%%%%%%%%%%%%%%%%%%%%%%%%%%%%%%%%%%%%%%
%% Present an original research topic that you would like to pursue in graduate school.
%% Describe the research idea, your general approach, as well as any unique resources that may be needed for accomplishing the research goal (i.e., access to national facilities or collections, collaborations, overseas work, etc.)
%% You may choose to include important literature citations.
%% Address the potential of the research to advance knowledge and understanding within science as well as the potential for broader impacts on society.
%% The research discussed must be in a field listed in the Solicitation (Section X, Fields of Study).
%% Page Limit - 2 pages
%%%%%%%%%%%%%%%%%%%%%%%%%%%%%%%%%%%%%%%%%%%%%%%%%%%%%%%%%%%%%%%%%%%%%%%%%%%%%%%%

\hdr{DRAFT: NSF Research Proposal}{Ben Greenman}

The first step in designing software is to pick the language best suited for the task.
A single programming language, however, is rarely sufficient to meet the needs of a growing project.
For example, Facebook began with a PHP script, but the now multi-billion dollar website currently includes code from at least 6 other languages, including C and Erlang.\footnote{http://www.quora.com/What-programming-languages-are-used-at-Facebook}
Conversely, Jane Street Capital was mostly Visual Basic and Excel before one aspiring employee began rewriting the core trading software in OCaml.\footnote{http://www.pcworld.idg.com.au/article/547567/won\_t\_believe\_what\_programming\_language\_wall\_street\_firm\_uses/}
This phenomena is widespread, and it arises because software projects encompass a spectrum of \emph{changing} goals but each modern language excels at only one specific class of problems.

My goal is to build a \emph{full spectrum programming language} that can accomodate a range of tasks, from high-assurance kernel software to dynamic scripts.
More importantly, this language will guarantee the interaction of its component languages.
Type system invariants will be preserved during interactions with untyped modules, and scripts that grow into full-fledged programs will be straightforward to port into typed or even dependently typed code.

\hdr{Research Proposal}{}

Full-spectrum programming is not a novel idea; in fact, the research community has been working on verifying multi-language systems for the past decade.
Pioneering work by Matthews and Findler studied a combined language of Scheme and ML~\cite{matthews2007operational}, and more recently Osera et.~al examined the interaction of simple and dependent types~\cite{osera}.
Other projects have explored topics like certified foreign-function interfaces~\cite{furr2005checking} and safe interoperability of untyped code with Java~\cite{gray2005fine}.
The time has come for a full-spectrum language that unifies these efforts.

My proposal is to study the interaction between dependently-typed and (un)typed code, building on the gradual typing work done by Sam Tobin-Hochstadt while at Northeastern University~\cite{tobin2010typed}.
Sam gave a full proof-of-concept for safe transitions between untyped and simply-typed modules.
I plan to extend his system with a dependently-typed language that permits all the formal verification of Coq or Agda, but moreover ensures safe operation with unverified languages similar to ML or C.
The feedback loop for realizing this plan will be to design a theoretical framework\textemdash initially I plan to use contracts to encode invariants like Coq proofs or parametricity requirements\textemdash and then implement it within the Racket ecosystem, which offers an extensible core language and diverse libraries to translate.

My progress will be directed by the group of experienced faculty at Northeastern University.
Matthias Felleisen has spent the past seven years working on Typed Racket.
Amal Ahmed perfected the proof method of step-indexed logical relations and has been using them to prove groundbreaking results on single and multi language systems.
Jan Vitek has been actively measuring gradually-typed languages and bridging the gap between academia and industry.
I am extremely lucky to have found such diverse mentors at one institution.

\newpage

\hdr{Broader Impacts}{}

The two motivating examples listed in my introduction are interesting, but relatively benign.
Much more compelling reasons for building a full spectrum language are the Swedish Pension system, which consists of 320,000 lines of Perl,\footnote{https://www.cs.purdue.edu/homes/jv/talks/dls09.pdf} and the HACMS project, which seeks to provide a simple interface for programming Unmanned Autonomous Vehicles (UAVs).\footnote{http://www.darpa.mil/Our\_Work/I2O/Programs/High-Assurance\_Cyber\_Military\_Systems\_(HACMS).aspx}

The pension system grew from a script that saved the country when a contractor failed to deliver, but now represents a sad pile of legacy code.
Sweden desperately needs a plan to certify individual modules and protect its retirees but a brute-force rewrite is not an option.
Conversely, the High-Assurance Cyber Military Systems (HACMS) project is seeking a solution for certified de-classification.
To create a simple API to their highly-sensitive machines, they essentially need a method of linking scripts to dependently typed programs.
These are precisely the issues my research will investigate.

Beyond its applications to real-world systems, a full-spectrum language is ideal for teaching programming concepts to novices because it offers smooth transitions from beginning to advanced topics.
I hope to introduce my full spectrum language to the Bootstrap program, an afterschool initiative with strong ties to Northeastern that teaches algebra to middle school students.\footnote{http://www.bootstrapworld.org}
Introducing dependent types in the Bootstrap program would allow a natural transition from algebra to logic and proofs.
Dependent types allow extremely powerful reasoning about programs, enough for students to write lemmas and theorems they might find in a textbook.
Being able to rewrite theorems as programs will help students better understand the ideas they see on the blackboard and improve their formal reasoning skills.
Thus a full-spectrum language is the key to a fully comprehensive program for mathematics education.

\vfill{}
\renewcommand{\section}[2]{}
\begin{multicols}{2}
\footnotesize
\bibliographystyle{plain}
\bibliography{nsf}
\end{multicols}

\end{document}
