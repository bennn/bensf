\documentclass{article}

\usepackage{fullpage}
\usepackage{palatino}
\usepackage{multicol}

\newcommand{\mono}[1]{\textbf{\texttt{#1}}}

\renewcommand\maketitle{
{\flushleft{\textsf{%
Ben Greenman
\hfill
DRAFT: NSF Research Proposal
}}\\
\hrulefill}
}
\begin{document}
\maketitle

%% %% this is pretty, but really it's just trash
%% Research in computer science\footnote{or for that matter, mathematics, biology, philosophy} seeks to build increasingly more sophisticated models for understanding and reasoning about real-world phenomena.
%% Successful models both take meaning from and influence the outside world.
%% In the subfield of programming languages, this effort has been extremely successful\textemdash we can now prove complex safety and behavioral properties about a program before running it, and abstractions like modules and objects give industry practitioners crucial tools to build intricate systems\textemdash but we are still missing the ``holy grail'' of full-spectrum programming languages.

%% A full-spectrum programming language is (in theory) a language for working with and expressing relationships between arbitrary programming languages.
%% %% We see hints of this in tools like Ott or Lem
%% %% It is a language of languages, if you will, similar to the set of all sets or category of categories
%% On one end, a full-spectrum language provides a launchpad for any application.
%% Developers may begin coding the language they are the most familiar with or effective in, then as the scope or goals of the project changes, they can easily migrate to a more suitable source language without incurring a huge engineering cost.
%% %% See for example Twitter's growing pains
%% Conversely, using a full-spectrum language as the target for compilation gives the ideal setting for reasoning about compiler correctness and proving behavioral properties.
%% No matter where a compiled program component originated, we can make guarantees about how it will interoperate with the code it is linked with.
%% If we as a research community care about helping creators to build scalable programs or about showing that the source-level guarantees we have worked to prove actually hold when the program is compiled, linked, and run, we need full-spectrum languages.

%% In fact, a full-spectrum language has meaning in terms of proofs, performance, and people: the biggest three outcomes we care about.
%% - People: uniform source
%% - Proofs: uniform target
%% - Performance: uniform backend

%% \subsubsection{About Me}
%% \subsubsection{What}
Programming languages are aptly characterized as tools; each has a few strengths and a few weaknesses, but as a whole the toolbox of modern-day languages helps developers realize a wide spectrum of ideas.
From low-level C to well typed ML to proof-checked Coq, there is a language suited for nearly every idea a person could dream of (and a research effort addressing those yet to be!).
%% 'live in isolation' -> are easy to contain | 
However, both ideas and software projects tend to grow and the reality of this ``language for every task'' mantra is that programs meeting a spectrum of needs must be spread over a spectrum of languages.
This fragmentation is unfortunately the best we can do; as a research community, we currently know very little about safe and effective language interoperability.

I envision a future where full-spectrum ideas and applications are served by correspondingly full-spectrum languages; tools that not only support the range of styles a developer needs but also mediate the interactions between them.
Should a programmer write a small, dynamically-typed script that links to a dependently typed package she found on GitHub, this interaction should both be immediate possible from the developer's perspective and proven sound by the encompassing language.
%% This is a simple request.

More formally, a full-spectrum language is a programming language whose source files may be compiled subject to a variety of correctness or well-formedness criterion depending on the syntax used within each source file.
One can imagine a language of languages\textemdash for example, a language with one sub-syntax for untyped programs and another sub-syntax for typed programs\textemdash but a full-spectrum language must be more than the sum of its parts.
It must additionally reason about the interaction between its component languages.
Any type soundness guarantees would need to hold even when typed modules interact with untyped modules, and any formally-verified proofs must be preserved despite interaction with code that has not been admitted by the mechanized theorem prover.

This is a simple request, but no language to date comes very close to satisfying it.
That said, there are glimpses of full-spectrum languages in foreign function interfaces~\cite{furr2005checking}, in gradually typed languages such as Grace\footnote{http://gracelang.org/} and Dart~\footnote{https://www.dartlang.org/}; and in research efforts that consider the interoperation of two or more languages~\cite{osera, gray2005fine, anand2014towards, perconti2014verifying}.
Although there has not yet been a formal effort towards building a full-spectrum language, we see a number of ways to make progress on this issue.

The first step is to build a core calculus of interoperability.
At present I am working on combining Matthews and Findler's calculus of Scheme and ML~\cite{matthews2007operational} with Osera et.~al's framework of simply-typed and dependently-typed lambda calculii~\cite{osera}.
The crucial part of this core calculus will be the boundaries separating components.
My initial hope was to encode these in a type system and verify interactions statically, but I now believe the dynamic power of contracts is necessary to describe correct and expressive boundaries.
Moreover, the numbers presented in Findler's ICFP 2014 keynote give evidence that contracts have an infectious nature, and as such may be essential for smoothly carrying entire modules across boundaries.
But regardless of the precise balance between contracts and types, there are three potential research topics that will improve the quality of these boundaries: space efficient contracts that implement the theories of Greenberg~\cite{greenberg2015space} and Herman et.~al~\cite{herman2010space}, type systems that can encode more of the diverse specifications allowed in contracts, and graceful transitions from contracts to types.
Regarding that final point, Greenberg's dissertation notes that over 80\% of contracts in the Racket codebase, one of the biggest users of contracts, only recover simple types~\cite{greenberg2013manifest}.
There is likely potential for automated or tool-assisted migration from these contracted modules to Typed Racket modules.

Second, I will need to build a language implementing this core calculus and test its efficacy.
My proposed venue for prototyping is the Racket language, which is developed largely at Northeastern University.
%% TODO userbase numbers
Racket has the advantages of a strong infrastructure and sizeable userbase; moreover, Racket already supports transitions for untyped Racket to the sister language Typed Racket so it is the ideal language to experiment with a broader spectrum of ideas.

Lastly, a full-spectrum language must be capable of accomodating new paradigms as they appear.
To meet this more nebulous goal, I plan on extending the spectrum of the core calculus with a few select components and using the experience to build a generalized algorithm for adding to the spectrum.
This algorithm could be as simple as a Coq script with well-documented holes, but it must be understandable such that a proprietary team is able to extend their codebase with an in-house language and verify the interactions of this new language with existing code.

This is a large agenda, but there are a similarly many incentives for building a full-spectrum language.
Foremost among these is the potential to help people currently using or hoping to learn a programming language.
One of the most frustrating aspects of programming is that the ideas one may express are limited by the language one works in.
Trying to write a UI in Agda is a bit like %attempting to write literature in the fictional language Newspeak of Orwell's \emph{1984}~\cite{orwell2009nineteen} or
using a spoon to paint a house.
In contrast, a full-spectrum language will allow developers to easily switch between languages well-suited for various tasks.
This impact will be strongly felt by teams working on large software projects: instead of being ingrained in one language, engineers may port one module into a more expressive component language without changing any other existing code.
Furthermore, the interaction between these components will be verified by the language.
This means the end of loosely-coupled multi-language projects, and it also defends against malware that might have leveraged some unchecked boundary in the code.

From the perspective of students, a full-spectrum language will make learning significantly easier.
In my own experience, one of the hardest parts about taking a new course was learning a new language on top of the new concepts we were supposed to grok.
Although I am grateful for the experience, adults who are seeking to enter the software engineering workforce cannot afford this luxury.
Imagine an unemployed professional trying to change careers, a mother seeking part-time employment, or an ex-convict looking to start a new life.
There are jobs waiting for these people; in fact, Facebook and Apple are now incentivizing women to delay motherhood rather than leave the workforce.\footnote{http://www.nbcnews.com/news/us-news/perk-facebook-apple-now-pay-women-freeze-eggs-n225011}
A full-spectrum language makes teaching easy.
Students can begin with a small, dynamically-typed core language.
As they master basic concepts, they can gradually turn to other elements of the spectrum depending on their personal goals.
A future web developer can study the contract system and learn how client and server code interacts.
An analyst can study how type systems help optimize numeric computations.
There is real potential for outreach.
%% bootstrap?

In addition to its benefits to language users, a full-spectrum language will help advance programming languages research.
There are a variety of efforts studying classes of program transformations.
At the moment, these efforts are crippled because the space of real-world programming languages is largely unstructured: links from one language to another are few.
By building a microcosm of the whole world in one full-spectrum language, we provide a baseline for other research.
Work on fully abstract compilers will be able to draw from the techniques the closed-world, full-spectrum language uses to ensure correctness.
Bidirectional transformations may also be influenced by the sort of connections we build.

%% Last but not least, the specialization and separation provided by a full-spectrum language will improve overall performance.
%% The compiler will be able to optimize more aggressively~\cite{leroy1994manifest} and of course the user will be writing code in the style best suited for the task.

%% This vision to build a full-spectrum language is really a no-brainer. % long overdue, etc etc
%% The need for larger and more stable software has created an immediate need for a full-spectrum language and research efforts into type systems and contract systems have given researchers the tools and vocabulary to begin making a serious effort.
%% The biggest open question (in my mind) if I have the imagination and perserverance to achieve it. % LOLOLOL

%% \subsubsection{Cool Ending}
Making discoveries and building inventions are just one part of a scientist's job.
Equally important, though not as glamorous, are the challenges of testing these new inventions, observing the world around us, and building connections between laboratory creations and real-world phenomena.
I hope that my proposal serves as a convincing argument that I will soon make progress towards building much-needed connections, and I hope that my future actions will lead to beautiful and useful results.

\vfill{}
\renewcommand{\section}[2]{}
\begin{multicols}{2}
\tiny
\bibliographystyle{plain}
\bibliography{nsf}
\end{multicols}

%% Dear Advisors,
%% I spent the evening dreaming by the window and finished a full draft.
%% With your permission, I'd like to send this very rough copy to my letter-writers.
%% Love Ben
\end{document}
