\documentclass[12pt]{article}

\usepackage[in]{fullpage}
\usepackage{multicol}

\newcommand{\mono}[1]{\textbf{\texttt{#1}}}

%% TODO merge header and name
\renewcommand\maketitle{
{\flushleft{\textsf{%
Ben Greenman
\hfill
DRAFT: NSF Research Proposal
}}\\
\hrulefill}
}
\begin{document}
\maketitle

%% %% this is pretty, but really it's just trash
%% Research in computer science\footnote{or for that matter, mathematics, biology, philosophy} seeks to build increasingly more sophisticated models for understanding and reasoning about real-world phenomena.
%% Successful models both take meaning from and influence the outside world.
%% In the subfield of programming languages, this effort has been extremely successful\textemdash we can now prove complex safety and behavioral properties about a program before running it, and abstractions like modules and objects give industry practitioners crucial tools to build intricate systems\textemdash but we are still missing the ``holy grail'' of full-spectrum programming languages.

%% A full-spectrum programming language is (in theory) a language for working with and expressing relationships between arbitrary programming languages.
%% %% We see hints of this in tools like Ott or Lem
%% %% It is a language of languages, if you will, similar to the set of all sets or category of categories
%% On one end, a full-spectrum language provides a launchpad for any application.
%% Developers may begin coding the language they are the most familiar with or effective in, then as the scope or goals of the project changes, they can easily migrate to a more suitable source language without incurring a huge engineering cost.
%% %% See for example Twitter's growing pains
%% Conversely, using a full-spectrum language as the target for compilation gives the ideal setting for reasoning about compiler correctness and proving behavioral properties.
%% No matter where a compiled program component originated, we can make guarantees about how it will interoperate with the code it is linked with.
%% If we as a research community care about helping creators to build scalable programs or about showing that the source-level guarantees we have worked to prove actually hold when the program is compiled, linked, and run, we need full-spectrum languages.

%% In fact, a full-spectrum language has meaning in terms of proofs, performance, and people: the biggest three outcomes we care about.
%% - People: uniform source
%% - Proofs: uniform target
%% - Performance: uniform backend

%% \subsubsection{About Me}
%% \subsubsection{What}
%% \vspace{-0.2cm}
%% \subsubsection*{Research Problem}
Programming languages are aptly characterized as tools; each has a few strengths and a few weaknesses, but as a whole the toolbox of modern-day languages helps developers realize a wide spectrum of ideas.
From low-level C to well typed ML to proof-checked Coq, there is a language suited for nearly every idea a person could dream of (and a research effort addressing those yet to be!).
%% 'live in isolation' -> are easy to contain | 
However, both ideas and software projects tend to grow and the reality of this ``language for every task'' mantra is that programs meeting a spectrum of needs must be spread over a spectrum of languages.
The Facebook website, for example, is powered by a mix of PHP, JavaScript, C, C++, Java, Python, and Erlang.\footnote{http://www.quora.com/What-programming-languages-are-used-at-Facebook}

Fragmentation of this sort is unfortunately the status quo; as a research community, we currently know very little about safe and effective language interoperability.
But I envision a future where full-spectrum ideas and applications are served by correspondingly full-spectrum languages; tools that not only support the range of styles a developer needs but also mediate the interactions between them.
%% Should a programmer write a small, dynamically-typed script that links to a dependently typed package she found on GitHub, this interaction should both be immediate possible from the developer's perspective and proven sound by the encompassing language.

More formally, a full-spectrum language is a programming language whose source files are compiled subject to a variety of correctness or well-formedness criterion depending on the syntax used within each source file.
One can imagine a language of languages\textemdash for example, with one syntax for untyped programs and another syntax for typed programs\textemdash but a full-spectrum language must be more than the sum of its components.
It must additionally provide boundaries and translations between each of these component languages.
Any type soundness guarantees proved in one component language would need to hold even when typed modules interact with untyped modules, and any formally-verified proofs must be preserved despite interaction with code that has not been admitted by a mechanized theorem prover.
These are exactly the properties one would need to gradually migrate from untyped to typed code, and to ensure that adding a new script to a project does not affect its overall correctness.

\vspace{-0.2cm}
\subsubsection*{General Approach}
There are glimpses of full-spectrum languages in foreign function interfaces~\cite{furr2005checking}, in gradually typed languages such as Grace\footnote{http://gracelang.org/} and Dart~\footnote{https://www.dartlang.org/}; and in research efforts that consider the interoperation of two or more languages~\cite{osera, gray2005fine, anand2014towards, perconti2014verifying}.
Hence although there has not yet been a formal effort towards building a full-spectrum language, I have a research plan that builds on prior work.

The first step is to build a core calculus of interoperability.
At present I am working on combining Matthews and Findler's calculus of Scheme and ML~\cite{matthews2007operational} with Osera et.~al's framework of simply-typed and dependently-typed lambda calculii~\cite{osera}.
The crucial part of this calculus will be the boundaries separating component languages; ideally, these boundaries will be verified by a static type system, but for now I am leveraging the dynamic power of contracts to describe boundaries.
This implies three directions for future research.
For one, we need more advanced type systems that can match the expressiveness of contracts.
Secondly, contracts have a large runtime cost, so we need to implement or extend the theories of Greenberg~\cite{greenberg2015space} and Herman et.~al~\cite{herman2010space} regarding space-efficient contracts.
Perhaps the best solution is to balance contracts and types in the vein of hybrid type-checking~\cite{knowles}; thus contracts, via their adaptive and infectious nature,\footnote{Findler's ICFP 2014 keynote showed that in the past 3 years over 3000 contracts have steadily appeared in the Racket ecosystem.} will demonstrate which inefficiencies type systems research should focus on solving.
Finally, there is potential for automated or tool-assisted migration from contracted modules to typed modules.
Studying the patterns of contract use in the Racket codebase, where over 80\% of contracts only recover simple types~\cite{greenberg2013manifest}, should provide concrete insights.

%% TODO don't like te intro
Second, I will build the language described by this model and test its efficacy.
My proposed venue for prototyping is the Racket language, which is developed largely at Northeastern University.
%% TODO userbase numbers
Racket has the advantages of a strong infrastructure, a sizeable userbase, and visionary developers.
Moreover, it already supports transitions from untyped Racket to the sister language Typed Racket, so it is the ideal language for experimenting with a broader spectrum of ideas.

Lastly, a full-spectrum language must be capable of accomodating new paradigms as they appear.
To achieve this goal, I plan on extending the spectrum of the core calculus with a few select components and using the experience to build a generalized algorithm for increasing the spectrum.
This algorithm could simply be a Coq script with well-documented holes, but it must be understandable such that a proprietary team is able to extend their codebase with an in-house language and verify the interactions of this new language with existing code.
Whether or not my prototype grows into a widely-used language, it will be an informative microcosm of our multi-language world.

\vspace{-0.2cm}
\subsubsection*{Broader Impacts}

A full-spectrum language is more than a flexible palette for engineers or a much-needed research experiment.
It also provides the ideal environment for learning a new language.
Beginning students may start learning in a small, dynamically typed language with only core language constructs.
As they grow more proficient, students can slowly learn more advanced features.
Unlike today, where exposing students to new ideas and paradigms often entails teaching them a brand new language, using a full-spectrum language keeps the focus on the programming concepts.

This approach has already been proven successful by the Bootstrap afterschool program, which teaches functional programming to underprivileged children in Boston, Harlem, and Chicago.\footnote{http://www.bootstrapworld.org/}
Bootstrap uses a series of educational languages precisely because its founders realized that ``no off-the-shelf programming language is suitable for novices''~\cite{felleisen2010teachscheme}.
Students gradually transition from a beginning language with only functions and conditionals to an ``advanced'' ML-like language.
However, the spectrum ends here, at ML.

As a graduate student, I hope to expand the languages taught in the bootstrap program to include type systems, object-oriented programming, scientific computing, and web design.
Furthermore, I want to expand the target audience of Bootstrap to include adults seeking to enter the tech industry.
One way to achieve this would be an afterschool program for students and their parents, so that stay-at-home adults can learn skills useful in today's economy.
Another option would be a completely new program for adults only, taught in the evenings or given as an online course.
We could also host outreach programs through prisons or the military, thus helping ease those adults' transition back into their communities.
At any rate, tech companies are desperately seeking talent\footnote{http://www.nbcnews.com/news/us-news/perk-facebook-apple-now-pay-women-freeze-eggs-n225011} and a full-spectrum language will provide the necessary groundwork for an educational program that quickly teaches the skills these companies will pay for.
%% TODO should likely pick a single angle and run.

%% This is a large agenda, but there are a similarly many incentives for building a full-spectrum language.
%% Foremost among these is the potential to help people currently using or hoping to learn a programming language.
%% One of the most frustrating aspects of programming is that the ideas one may express are limited by the language one works in.
%% Trying to write a UI in Agda is a bit like %attempting to write literature in the fictional language Newspeak of Orwell's \emph{1984}~\cite{orwell2009nineteen} or
%% using a spoon to paint a house.
%% In contrast, a full-spectrum language will allow developers to easily switch between languages well-suited for various tasks.
%% This impact will be strongly felt by teams working on large software projects: instead of being ingrained in one language, engineers may port one module into a more expressive component language without changing any other existing code.
%% Furthermore, the interaction between these components will be verified by the language.
%% This means the end of loosely-coupled multi-language projects, and it also defends against malware that might have leveraged some unchecked boundary in the code.

%% In addition to its benefits to language users, a full-spectrum language will help advance programming languages research.
%% There are a variety of efforts studying classes of program transformations.
%% At the moment, these efforts are crippled because the space of real-world programming languages is largely unstructured: links from one language to another are few.
%% By building a microcosm of the whole world in one full-spectrum language, we provide a baseline for other research.
%% Work on fully abstract compilers will be able to draw from the techniques the closed-world, full-spectrum language uses to ensure correctness.
%% Bidirectional transformations may also be influenced by the sort of connections we build.

%% Last but not least, the specialization and separation provided by a full-spectrum language will improve overall performance.
%% The compiler will be able to optimize more aggressively~\cite{leroy1994manifest} and of course the user will be writing code in the style best suited for the task.

%% This vision to build a full-spectrum language is really a no-brainer. % long overdue, etc etc
%% The need for larger and more stable software has created an immediate need for a full-spectrum language and research efforts into type systems and contract systems have given researchers the tools and vocabulary to begin making a serious effort.
%% The biggest open question (in my mind) if I have the imagination and perserverance to achieve it. % LOLOLOL

\vspace{-0.2cm}
\subsubsection*{Conclusion}
Making discoveries and building inventions are just one part of a scientist's job.
Equally important, though not as glamorous, are the challenges of testing these new inventions, observing the world around us, and building connections between laboratory creations and real-world phenomena.
I hope that my proposal serves as a convincing argument that I will soon make progress towards building much-needed connections, and I hope that my future actions will lead to beautiful and useful results.


\renewcommand{\section}[2]{}
\begin{multicols}{2}
\footnotesize
\bibliographystyle{plain}
\bibliography{nsf}
\end{multicols}

\end{document}
